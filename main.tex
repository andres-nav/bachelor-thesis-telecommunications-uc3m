\documentclass[oneside, 12pt, a4paper]{book}

\errorcontextlines=5 % Adjust the number to show more or fewer lines

\makeatletter
\def\input@path{{./templates/}}
\usepackage{thesis_uc3m}
\usepackage{eurosym}
\usepackage{longtable}
\graphicspath{{./figures} {./templates/figures}}
\makeatother

\title{Building a Scalable Parking Management Solution for Multi-Community Deployment}
\author{Andrés Navarro Pedregal}
\degree{Data Science and Engineering and Telecommunication Technologies Engineering}
\graduationyear{2024~--~2025}
\supervisor{David Larrabeiti López}
\placeandyear{Leganés, 2025 \\ Madrid, Spain}

\hypersetup{
  pdfauthor=Andres Navarro Pedregal,
  pdftitle=Final Thesis | Andres Navarro Pedregal | Telecommunication Technologies Engineering,
  pdfsubject=Final Thesis,
  pdfkeywords={Final Thesis, Andres Navarro Pedregal, Telecommunication Technologies Engineering},
}

\addbibresource{./bibliography.bib} % load the bibliography file

\loadglsentries{acronyms} % load the acronyms file

\begin{document}

% \setcounter{tocdepth}{0} % Show only parts and chapters in the table of contents

\frontmatter
\maketitle

\blankpage%
\chapter*{Abstract}

This bachelor thesis presents the design, implementation, and deployment of a distributed parking management system tailored for multi-community environments. The project addresses the growing challenges of urban parking management through an innovative approach that combines edge computing with cloud services.

The system employs a distributed architecture where each community operates autonomously while maintaining centralized oversight. Key components include license plate recognition using machine learning, automated access control, and a web-based management interface. The solution utilizes NVIDIA Jetson Nano devices for edge processing, custom relay boards for hardware control, and AWS cloud infrastructure for centralized management.

The system has been successfully deployed across ten communities in Valencia, Spain, demonstrating its scalability and reliability in real-world conditions. The implementation incorporates robust security measures, including encrypted databases, secure network communications via ZeroTier, and SMS-based user authentication. Environmental challenges, such as the Dana weather event in September 2024, led to system improvements in hardware protection and reliability.

This work contributes to the field of smart city infrastructure by providing a practical, scalable solution for community parking management that balances autonomy with centralized control, setting a foundation for future expansion and enhancement of urban parking solutions.

\textbf{Keywords:} parking management, distributed systems, edge computing, cloud computing, license plate recognition, IoT, smart cities

\blankpage%

\chapter*{Acknowledgments}
\begingroup
\let\clearpage\relax % This temporarily disables \clearpage so the Agradecimientos chapter is on the same page

First and foremost, I would like to express my sincere gratitude to my supervisor, Dr. David Larrabeiti López, for the opportunity to work with him on this project. His guidance, expertise, and support have been invaluable throughout the development of this thesis.

My heartfelt appreciation goes to the Telematics Department of Universidad Carlos III de Madrid, especially to my supervisor again and Dr. José Alberto Hernández Gutiérrez for their unwavering support since the beginning of my academic journey. I am grateful to all members of the department for creating a nurturing and stimulating environment for research and learning.

To ATES, specially Jose Peris for their support and collaboration in the development of the parking management system.

Finally, I would like to thank my colleague Álvaro Ramajo for his collaboration and support throughout this process. Your camaraderie and intellectual contributions have been instrumental in the completion of this thesis.

To all those mentioned and to the many others who have supported me along the way, thank you for your encouragement, patience, and belief in my abilities. This work would not have been possible without you.

\chapter*{Agradecimientos}

En primer lugar, quiero expresar mi más sincero agradecimiento a mi supervisor, el David Larrabeiti López, por la oportunidad de trabajar con él en este proyecto. Su orientación, experiencia y apoyo han sido invaluables durante todo el desarrollo de esta tesis.

Mi más profundo aprecio va para el Departamento de Telemática de la Universidad Carlos III de Madrid, especialmente otra vez a mi supervisor y Dr. José Alberto Hernández Gutiérrez por el apoyo inquebrantable desde el inicio de mi trayectoria académica. Estoy agradecido a todos los miembros del departamento por crear un ambiente estimulante y enriquecedor para la investigación y el aprendizaje.

A ATES, especialmente a Jose Peris por su apoyo y colaboración en el desarrollo de este sistema.

Finalmente, me gustaría agradecer a mi colega Álvaro Ramajo por su colaboración y apoyo durante todo este proceso. Tu camaradería y contribuciones intelectuales han sido fundamentales para la realización de esta tesis.

A todos los mencionados y a los muchos otros que me han apoyado en el camino, gracias por su aliento, paciencia y fe en mis capacidades. Este trabajo no habría sido posible sin ustedes.

\endgroup

\blankpage%

\renewcommand{\contentsname}{Table of Contents}
\tableofcontents

\blankpage%

\listoffigures

\blankpage%

\listoftables

\blankpage%

\printglossary[type=\acronymtype,style=long, title=LIST OF ACROYNMS]

\blankpage%

\mainmatter%

\oldpart{Introduction} % This is the first part, so it is not preceded by a new page

\chapter{Motivation}\label{ch:motivation}

The rapid growth of urban populations and the corresponding increase in vehicle ownership have significantly intensified the challenges associated with parking management in cities worldwide. For instance, the dramatic rise in car usage in Madrid has substantially contributed to heightened levels of environmental pollution \autocite{environmental_impact_madrid_central}. Despite the number of parking spaces in Spain remaining stable between 2014 and 2020 \autocite{urban_mobility_trends}, the escalating demand for parking has resulted in higher costs, prolonged search times, increased traffic congestion, and elevated urban pollution levels.

The inadequacies of traditional, manual \glspl{pms} have further highlighted the need for innovative solutions in this domain. Studies highlight the importance of user-friendly, secure, and reliable systems, with these factors playing a crucial role in influencing parking behavior and preferences \autocite{parking_choices}. Consequently, the development of effective \glspl{pms} remains an essential area of research, particularly as urbanization continues to accelerate.

Advancements in technology provide new opportunities to address these challenges. The proliferation of internet-connected devices, growing by 20\% annually \autocite{iot_growth}, has catalyzed the emergence of \gls{iot}-based \glspl{pms}. These systems aim to mitigate traffic congestion and reduce urban pollution while promoting sustainable and efficient urban mobility. By leveraging real-time data and automation, modern \glspl{pms} offer transformative potential for urban planning and management.

Integrating these advanced systems into urban environments offers multiple benefits, including decreased traffic congestion, reduced pollution, enhanced security, and an improved quality of life for residents. Additionally, \glspl{pms} enable automated parking space management, making them adaptable for a wide range of applications, including urban centers, residential communities, and commercial buildings.

Despite these advancements, the widespread adoption of \glspl{pms} in Spain has faced significant obstacles. Many existing systems rely heavily on human intervention, leading to delays, insufficient information dissemination, and limited control over parking space availability. This dependence on manual operations not only impairs system efficiency but also increases operational costs.

Emerging technologies have inspired the development of innovative \glspl{pms}, such as RFID-based systems \autocite{rfid_smart_parking_management_system}, \gls{iot}-enabled solutions \autocite{development_smart_parking_management_system}, and intelligent parking systems employing image processing techniques \autocite{intelligent_parking_system_image_processing}. However, these systems frequently rely on centralized architectures, which are prone to scalability and reliability issues, particularly during service disruptions. Moreover, many current solutions lack the flexibility needed to accommodate the specific requirements of diverse user groups, limiting their broader applicability.

Addressing these limitations, the primary objective of this project is to design and implement a fully distributed and autonomous \gls{pms} that overcomes the challenges inherent in existing systems. This new approach emphasizes scalability, reliability, and adaptability, with a particular focus on meeting the demands of next-generation smart cities. Through this work, the project aims to contribute to the development of sustainable, efficient, and user-centric parking solutions for urban environments.

\todo{add a graph or somehting to showcase the importance}


\chapter{Statement of the problem}\label{ch:statement_of_problem}

\todo{write the statementof the problem}

\chapter{Objectives}\label{ch:objectives}

The primary objective of this bachelor thesis is to design and implement a fully distributed parking management system tailored for the next generation of smart cities. This project aims to address the inefficiencies and challenges inherent in current parking management systems through a distributed approach that leverages modern technologies. The objective of this project can be broken down into the following specific goals:

\begin{itemize}
  \item Conduct a comprehensive study of existing parking management systems, identifying their main problems and inefficiencies. This involves understanding user requirements, analyzing the technologies employed, and evaluating system effectiveness.

  \item Analyze the technologies currently used in parking management systems to determine their suitability for a distributed architecture. This includes examining sensors, \gls{iot} devices, communication protocols, and data processing methods to identify the most suitable technologies for the proposed system.

  \item Design the overall infrastructure of the distributed parking management system. This encompasses defining the system architecture, selecting appropriate technologies, and developing detailed design specifications to ensure a scalable, secure, and user-friendly system.

  \item Develop the system by adhering to a structured methodology that includes phases of planning, design, implementation, testing, deployment, and maintenance. Each phase will follow best practices to ensure the system’s robustness, efficiency, and reliability.

  \item Evaluate the implemented system based on various criteria such as performance, scalability, security, usability, reliability, availability, and cost. This comprehensive analysis will help in assessing the effectiveness of the system and identifying areas for improvement.

  \itemDeploy and test the proposed system in a real-world scenario, such as a city center with high vehicle density and limited parking spaces, to validate its effectiveness and identify potential issues.
\end{itemize}

By achieving these objectives, the thesis aims to contribute to the advancement of smart city technologies by providing a scalable, secure, and user-friendly parking management solution. The distributed nature of the proposed system is expected to enhance its performance and reliability, making it a viable option for modern urban environments.

\chapter{Document Structure}\label{ch:document_structure}

This thesis is organized into six main parts, each addressing specific aspects of developing a scalable parking management solution for multi-community deployment.

Theoretical Background, \cref{part:theoretical_background}, provides essential knowledge for understanding the technological foundations of the project. It explores cloud computing concepts, including deployment models and service types crucial for system scalability. This part also covers Internet of Things (IoT) technologies, focusing on their architecture, benefits, and challenges in modern applications.

State of the Art, \cref{part:state_of_the_art}, examines the evolution and current state of parking management systems. It traces the historical development of these systems from early mechanical solutions to modern digital implementations. The section on modern trends explores current technologies and approaches in parking management, providing context for the project's innovations.

Methodology, \cref{part:methodology}, forms the core of the thesis, detailing the system's development process. It begins with comprehensive requirements analysis, covering both user and system needs. The design section outlines the system architecture and hardware selections, while the implementation section details the practical realization of the system, including both community-level and cloud components.

Results, \cref{part:results}, presents the outcomes of the project through testing and deployment phases. The testing section covers individual component validation and integrated system testing, while the deployment section discusses the real-world implementation across multiple communities and the challenges encountered.

Conclusions, \cref{part:conclusions}, summarizes the project's achievements and insights. It includes future work recommendations. This part provides a comprehensive overview of the project's implications and potential future developments.

\chapter{Methodological Framework}\label{ch:methodology_approach}

The methodological framework employed in this thesis is grounded in the V-model as established by the \gls{incose} \autocite{INCOSE2015} for project development. The V-model offers a rigorous and structured method that ensures all project facets are considered, facilitating timely and budget-compliant completion. This is achieved through a comprehensive development process, enabling clear validation and verification of initial requirements at every stage.

The methodology is segmented into seven key components which can be summarized as follows:

\begin{enumerate}
  \item \textbf{Identification of User Requirements}: A detailed analysis of the problem statement is conducted to identify the primary issues and potential solutions. Moreover, the user requirements are defined to ensure that the proposed solution aligns with the objectives of the project.

  \item \textbf{System Design}: The system architecture is developed based on the user requirements, ensuring that the proposed solution is feasible and aligns with the project's objectives. This phase includes a detailed overview of the system components and their interconnections. Requirements are formulated to satisfy the previously defined solution requirements. This phase includes a high-level overview of the components of the proposed solution, the justification for their selection, and the interconnections among them.

  \item \textbf{Component Design}: Building upon the high-level architecture of the solution, a more detailed approach is outlined for each component, taking into account their specific power and data transmission needs. This culminates in a comprehensive architecture of the solution. Furthermore, a detailed overview of the components is provided, including the rationale for their selection and the interconnections among them.

  \item \textbf{Implementation}: The proposed solution is implemented and manufactured utilizing available tools while simultaneously integrating the necessary electrical components. This phase includes a detailed description of the implementation process, including the tools and materials used, as well as the integration of electrical components. The development of software and hardware components is also detailed.

  \item \textbf{Component Testing}: The functionality of each component is verified in a standalone mode, with detailed information provided regarding the verification process.

  \item \textbf{System Testing}: The methodology for conducting flight tests and subsequent analyses is elaborated. System integration is performed by assessing communication between module pairs to ensure that data can be transmitted freely and utilized effectively.

  \item \textbf{Acceptance Testing}: Validation of the initial requirements is conducted to confirm that all solution requirements have been met. This phase also includes preparations for potential future enhancements.
\end{enumerate}

\begin{figure}
  \includegraphics{v_model_methodology.png}
  \caption{Methodological framework based on the V-model from \glsentryshort{incose} with the different stages of the project development process\autocite{ruddle2020vmodel}}\label{fig:v_model_methodology}
\end{figure}


Moreover, a graphical representation of the V-model is provided in \cref{fig:v_model_methodology} to illustrate the methodology's structure and the relationship between the various stages.

% Local Variables:
% jinx-local-words: "incose"
% End:


\oldpart{Theoretical Background}\label{part:theoretical_background}

\chapter{Cloud Computing}\label{ch:cloud_computing}

Cloud computing is a technological paradigm that enables the delivery of computing services, including servers, storage, databases, networking, software, and analytics, over the internet. This model shifts the management of physical infrastructure and resources from on-premises systems to remote data centers managed by cloud service providers. Cloud computing is fundamental to modern digital ecosystems, supporting a wide range of applications from personal use to large-scale enterprise operations.

\section{Definition and Key Concepts}

Cloud computing leverages a network of remote servers hosted on the internet to store, manage, and process data, rather than relying on local computers or private data centers. It offers resources on-demand, enabling users to scale operations according to their needs. The essential characteristics of cloud computing include:

\begin{itemize}
	\item \textbf{On-Demand Self-Service:} Users can access computing resources as needed without requiring human interaction with service providers.
	\item \textbf{Broad Network Access:} Services are accessible over a network, typically the internet, using various devices such as smartphones, tablets, and laptops.
	\item \textbf{Resource Pooling:} Resources are pooled to serve multiple users, ensuring efficiency and scalability.
	\item \textbf{Rapid Elasticity:} Resources can be elastically provisioned and released, enabling scalability according to demand.
	\item \textbf{Measured Service:} Cloud systems automatically control and optimize resource use, charging users based on their consumption.
\end{itemize}

\section{Advantages of Cloud Computing}

The adoption of cloud computing offers several advantages that have contributed to its widespread popularity. One of the most significant benefits is its scalability and flexibility. Cloud computing enables organizations to adjust their resource usage dynamically based on real-time needs, eliminating the need for over-provisioning and supporting workloads of varying demands effectively. This adaptability ensures that resources are utilized efficiently, minimizing waste and optimizing performance.

Cost efficiency is another critical advantage. By using cloud services, organizations can avoid the significant capital expenditures associated with purchasing and maintaining hardware and software. Instead, they incur operational costs only for the resources they consume, making cloud computing an economically viable solution for businesses of all sizes. Additionally, cloud computing enhances accessibility and collaboration. Since cloud-based applications and services are accessible from anywhere with an internet connection, distributed teams can collaborate seamlessly, a feature that has become indispensable in today’s increasingly globalized and remote work environments.

Reliability is also a key strength of cloud computing. Leading cloud providers ensure high levels of availability by deploying redundant systems across geographically dispersed data centers, thereby minimizing downtime and enhancing disaster recovery capabilities. Furthermore, security is a paramount concern addressed by cloud providers. They invest significantly in advanced measures such as encryption, firewalls, and regular security audits, ensuring robust protection against cyber threats and unauthorized access. These combined advantages make cloud computing a transformative technology, supporting innovation and operational efficiency in diverse sectors.

\section{Types of Cloud Computing}

Cloud computing services are categorized into different types based on their deployment models and the services they provide. These distinctions allow organizations to select a cloud strategy that aligns with their specific requirements.

\subsection{Deployment Models}

\begin{itemize}
	\item \textbf{Public Cloud:} A public cloud is owned and operated by third-party providers, delivering resources over the internet. Examples include \gls{aws}, Microsoft Azure, and Google Cloud Platform. It offers cost-effectiveness and scalability but may pose data sovereignty concerns for certain organizations.
	\item \textbf{Private Cloud:} A private cloud is used exclusively by a single organization. It offers greater control over data and infrastructure, making it suitable for industries with strict regulatory requirements.
	\item \textbf{Hybrid Cloud:} A hybrid cloud combines public and private cloud environments, enabling organizations to benefit from both scalability and control. This model is particularly useful for balancing workloads and maintaining sensitive data on-premises while leveraging the public cloud for other operations.
	\item \textbf{Multi-Cloud:} A multi-cloud strategy involves using multiple cloud providers to mitigate risks associated with vendor lock-in and enhance reliability and performance.
\end{itemize}

\subsection{Service Models}

Cloud computing services are typically offered in the following models:

\begin{itemize}
	\item \textbf{\gls{iaas}:} Provides virtualized computing resources over the internet, including servers, storage, and networking. Users manage operating systems and applications while the provider handles the infrastructure.
	\item \textbf{\gls{paas}:} Offers a platform that includes hardware, software, and development tools to build, test, and deploy applications. It abstracts the complexities of infrastructure management, enabling developers to focus on coding.
	\item \textbf{\gls{saas}:} Delivers software applications over the internet on a subscription basis. Users can access the software through a browser without worrying about installation or maintenance.
\end{itemize}

\chapter{Internet of Things}\label{iot}

The \gls{iot} is a paradigm that establishes a network of interconnected devices equipped with sensors, software, and communication technologies. This enables seamless interaction between the physical and digital realms, facilitating the collection, transmission, and processing of data. Through its ability to deliver enhanced functionality and automation, the \gls{iot} has numerous applications, particularly in smart city infrastructure. These applications span healthcare, transportation, energy management, and urban planning.

The main point of \gls{iot} is its capacity for interconnectivity, enabling devices to autonomously communicate and share data. Advanced sensing technologies augment this connectivity, capturing real-world metrics such as temperature, motion, or occupancy. The processing of the data is achieved through edge computing, which provides localized analysis, or cloud computing for centralized processing, tailored to meet latency and scalability demands. This processed data is subsequently integrated with actuators, enabling automation of tasks such as the management of parking space access without requiring human intervention.

\section{Architecture of IoT Systems}
The architecture of \gls{iot} systems is structured across four principal layers. The perception layer constitutes sensors and actuators, which interface directly with the physical environment to gather data and execute specific actions. Data transmission occurs within the network layer, leveraging communication protocols such as WiFi, Bluetooth, or cellular networks. The processing layer transforms raw data into actionable insights through edge or cloud computing methodologies. Finally, the application layer provides interfaces for end-users, ensuring an intuitive and accessible experience tailored to diverse functionalities.

\section{Benefits of IoT}
The adoption of \gls{iot} technologies confers significant advantages across multiple sectors, particularly in the realm of parking management systems. Real-time monitoring capabilities, enabled by sensors, deliver continuous updates on parking space availability, reducing search times and alleviating urban congestion. Automation, facilitated by actuators, streamlines operations such as access control and reservation management, minimizing human intervention. Moreover, \gls{iot}-driven data analytics empower urban planners with insights to optimize resource allocation and develop informed policies. The energy efficiency inherent in smart systems aligns with sustainability objectives by curbing unnecessary resource consumption.

\section{Challenges and Limitations}
The deployment of \gls{iot} systems presents several challenges. Scalability remains a critical concern, particularly in large-scale implementations like city-wide parking systems. The management of sensitive information collected by \gls{iot} devices necessitates data security and privacy measures to itigate risks. Interoperability challenges, stemming from disparate manufacturer standards and communication protocols, complicate system integration.



\oldpart{State of the art}\label{part:state_of_the_art}

\chapter{Historical Development}\label{ch:historical_development}

In the early 20th century, as urbanization accelerated and automobile ownership increased, the need for efficient parking solutions became apparent. The first significant step in this direction came in 1905 with the Garage Rue de Ponthieu \autocite{mcdonald2005parking} in Paris, designed by Auguste Perret. This multi-story concrete structure, see \cref{fig:garage_rue_de_ponthieu}, featured internal elevators to transport cars between floors, where attendants would manually park them. While not fully automated, this innovation laid the foundation for future parking systems by maximizing vertical space utilization.

The 1920s marked the beginning of more advanced parking solutions, particularly in major U.S. cities. The Paternoster system \autocite{paternoster2022}, resembling a Ferris wheel for cars, refer to \cref{fig:pater_noster_parking_system}, could accommodate eight vehicles in the space typically used for two. This period also saw the introduction of Kent Automatic Garages in New York in 1928, which employed an electric "parker" to lift and move cars to available spaces. These early automated systems represented significant progress in addressing the growing demand for parking in densely populated urban areas.

\begin{figure}
	\hfill
	\begin{subfigure}{0.45\textwidth}
		\includegraphics{garage_du_rue_ponthieu.jpg}
		\caption{Garage Rue de Ponthieu in Paris, 1905 \autocite{mcdonald2005parking}.}\label{fig:garage_rue_de_ponthieu}
	\end{subfigure}
	\hfill
	\begin{subfigure}{0.45\textwidth}
		\includegraphics{paternoster.png}
		\caption{Paternoster system, 1920s \autocite{paternoster2022}.}\label{fig:pater_noster_parking_system}
	\end{subfigure}
	\hfill

	\caption{Historical developments in parking management systems.}
\end{figure}

The 1930s brought another crucial development with the introduction of parking meters, revolutionizing on-street parking management. This innovation allowed cities to better regulate parking durations and generate revenue from public parking spaces. The subsequent decades, particularly the 1940s and 1950s, witnessed a surge in the development of automated parking systems in the United States, with designs such as Bowser, Pigeon Hole, and Roto Park \autocite{shoup2012cars} gaining popularity.

While interest in automated parking systems temporarily waned in the U.S. during the 1960s to 1980s, development continued in Europe and Asia. Notable advancements included the Auto Stacker system in London in 1961 and Wohr's Electromechanical Parking System Type 100 \autocite{hardingaps2021history} in Germany in 1962. During this period, Japan emerged as a leader in automated parking systems, with significant growth continuing into the 1990s.

The late 20th century marked the beginning of the digital revolution in parking management. The 1990s saw the introduction of parking guidance systems, which used sensors to monitor occupancy and provide real-time information to drivers. This technology significantly reduced the time and frustration associated with finding available parking spaces in large facilities.

The early 2000s ushered in a new era of smart parking solutions. In 2002, the first robotic parking garage in the United States opened in Hoboken, New Jersey \autocite{usatoday2007robotic}, showcasing the potential of fully automated parking facilities. This period also saw the rapid development of technologies such as license plate recognition systems, mobile parking apps for finding and reserving spaces, and IoT-based parking management systems.

In recent years, parking management systems have continued to evolve, incorporating advanced technologies such as artificial intelligence and machine learning for predictive parking analytics. Cloud-based parking management platforms have become increasingly common, offering scalable and flexible solutions for parking operators. Furthermore, the integration of parking systems with broader smart city initiatives and connected vehicle technologies is paving the way for more efficient and sustainable urban mobility solutions.

The evolution of parking management systems reflects broader technological trends, moving from mechanical solutions to digital, interconnected systems. Today's parking management solutions prioritize efficiency, user experience, and environmental sustainability, addressing not only the immediate needs of drivers and parking operators but also contributing to the broader goals of smart urban planning and reduced environmental impact.

As cities continue to grow and evolve, parking management systems will undoubtedly play a crucial role in shaping the future of urban mobility. The ongoing development of these systems promises to further optimize space utilization, reduce traffic congestion, and enhance the overall urban experience for residents and visitors alike.

\chapter{Modern Trends}\label{ch:modern_trends}

Automated parking management systems have evolved significantly to address the unique needs of residential communities. Currently, various types and technologies are employed in modern parking management solutions, focusing on their application in community settings. Sensor-based systems utilize advanced sensing technologies, such as ground sensors embedded in pavement and overhead sensors mounted above parking spaces, to monitor and manage parking spaces efficiently. These sensors provide real-time data on occupancy, enabling efficient space utilization and reducing search times for residents.

\gls{lpr} systems have gained popularity in community parking management, offering advantages such as automated access control, visitor management, and enforcement of parking regulations. \gls{rfid} technology provides another efficient solution, particularly beneficial for large residential complexes with multiple parking areas. Mobile application-integrated systems offer residents a user-friendly interface to interact with the parking system, providing features like real-time availability information, digital permits, and reservation capabilities.

Cloud-based management platforms have revolutionized parking management by offering centralized, scalable solutions with real-time data analytics, remote management capabilities, and enhanced security. Smart gate systems, integrated with other parking management technologies, form a crucial component of modern community parking solutions, enhancing security and ensuring smooth traffic flow.

The choice of parking management system for a residential community depends on factors such as the size of the community, budget constraints, and specific parking challenges. Often, a combination of these technologies is employed to create a comprehensive, efficient, and user-friendly parking management solution tailored to the unique needs of each community. As urban populations continue to grow and vehicle ownership increases, these advanced parking management systems will play an increasingly important role in optimizing space utilization, improving resident satisfaction, and contributing to more sustainable urban environments.


\oldpart{Methodology}\label{part:methodology}

\chapter{Requirements}\label{ch:requirements}

\todo{Write this chapter}

The parking management system developed for this project was designed to ensure that it met the needs of drivers, parking facility managers, and city administrators.
The system was designed to address the limitations of current parking management systems, focusing on enhancing scalability, reliability, and user adaptability.

For this purpose, different user requirements were identified, and system requirements were defined to meet these needs.
Furthermore, the system was designed to incorporate specific technical and functional specifications to ensure its effectiveness and efficiency.

\section{User Requirements}\label{sec:user_requirements}

The user requirements for the distributed parking management system were identified through a comprehensive analysis of the needs and preferences of drivers, parking facility managers, city administrators and users with disabilities.
These requirements were essential to ensure that the system was user-friendly, efficient, and aligned with the objectives of smart city initiatives \autocite{smart_cities_initiatives}.

The requirements of the primary users were as follows:

For drivers, the system needed to provide real-time information about available parking spaces to minimize search time.
This was crucial to reduce traffic congestion and pollution in urban areas.
Moreover, drivers expected an easy-to-use interface for quick navigation and more importantly, automatic functionality of the system without the need for human intervention.
That way, they did not have to worry about the availability of parking spaces and could focus on other tasks.

Parking facility managers required tools to monitor and manage their facilities efficiently, as well as access to detailed reports and analytics on parking usage patterns to optimize space utilization.
A notification system for intrusions such as unauthorized parking or security breaches was also essential to ensure the safety of the parking facilities.
Enhanced security measures, including surveillance and access control, were a must-have for them.

On the other hand, city administrators needed a system that could provide insights into parking demand and usage trends to inform urban planning decisions.
The system should support the integration of parking data with other smart city initiatives to enhance overall urban mobility and sustainability.
Moreover, city administrators required tools to monitor and minimize the environmental impact of parking facilities, such as reducing emissions and energy consumption.
The system should also comply with local regulations and standards for data privacy and security such as GDPR \autocite{gdpr}.

Finally, users with disabilities needed accessibility features such as voice commands, screen readers, and other assistive technologies to ensure that they could use the system effectively.
These features were essential to promote inclusivity and ensure that all users could benefit from the distributed parking management system.
Moreover, the system should be designed to accommodate users with different needs and preferences, ensuring a seamless user experience for everyone.

\section{System Requirements}\label{sec:system_requirements}

For the distributed parking management system to meet the user requirements, it had to incorporate specific system requirements.
These requirements were defined to ensure that the system was scalable, reliable, secure, and user-friendly, aligning with the objectives of smart city initiatives as well as the needs of drivers, parking facility managers, and city administrators.
The system requirements were essential to guide the design and development of the distributed parking management system, ensuring that it met the expectations of all stakeholders and delivered the desired outcomes.

The main system requirements were as follows:

For the real-time monitoring of parking space availability, the system needed to incorporate sensors and IoT devices to detect vehicle presence and occupancy.
These sensors had to be accurate, reliable, and cost-effective to provide up-to-date information on parking availability.

Reservation was another key requirement, allowing drivers to be assigned a specific number of parking spaces in advance to ensure that they could secure a spot when needed.
The reservation system had to be integrated with the monitoring system to ensure that drivers could find available spaces.

The system also needed to be self-sufficient, with automated gate and barrier control to regulate access to parking facilities.
It had to be capable of managing multiple parking facilities simultaneously, ensuring that drivers could access the system from different locations.
And it had to be able to function without human intervention, reducing the need for manual oversight.

Moreover, the system had to be scalable and flexible, with a distributed architecture that could handle large volumes of data and a growing number of users.
It had to support additional number of communities and parking facilities, ensuring that it could adapt to changing requirements.
For the system to be effective, modularity was essential, allowing for easy updates and integration of new features as needed.

Security and availability were critical requirements for the system, ensuring that the system was robust and resilient against cyber threats and service interruptions.
Interoperability was crucial to ensure compatibility with existing and future smart city infrastructure.

Finally, the user interface had to be intuitive and user-friendly, with mobile accessibility for drivers and web-based interfaces for facility managers and city administrators.
And accessibility features were needed to support users with disabilities, ensuring that the system was inclusive and accessible to all.

\chapter{Design}\label{ch:design}

This chapter outlines the architecture of the system design, detailing the different subsystems and components to meet the requirements detailed in \cref{ch:requirements}. The system designed is composed of two main components, a server, in charge of manageming the users, authenticating them, and relaying the information form the users, and the terminals, which are the brains of the system and hold the main control for each community. It is worth to note that while the terminal is deployed in every community, the server is only deployed once and can be used for every user in the system. The architecture of the system can be seen in \todo{add image with architecture}. Note that there are other complementary elementes that they will be later explained.

\todo{ add image of architecture and explain}

\section{Terminals}

The terminals are the core part of the deployment and is where all the processing is done to manage each community. In order to provide the necesary capabilities for the community such as opening the garage doors and detecting the cars the following components are needed.

\subsection{Cameras}

Two cameras per door to be able to detect the cars when they are close to the garage door. This cameras are the ones in charge of sending live video to be able to detect the cars and later identify the licence plates. The requirements for the cameras are that they are wether proof and can withstand heavy rain, can send live feed and be connected to a central computer, and have support for infrared detection for low light events such as inside garage parkings or night. Different models of cameras where studied but the final camera used for this project was the \todo{add camera} because of \todo{add reason}

\todo{add image of camera}

Moreover, even thought the camera has infrarred capabilities, the infrarred light of the camera is pretty weak. Therefore an anditional infrared led was used to increase the quality of the image at night. Also the good part of this camera is their relative low price point and that they can be powered with \gls{poe}, that is, they can be power and connected to the networking with just one cable.

\todo{add infrared image and description}

\subsection{Computer}

For the computer, the main requirements where that it would need to have the necessary computing power to run the detection algorithm to detect the different vehicles and licence plates for each camera and the price point is low enough as there will be one in every community. This restiction to be able to run the detection algorithm limits to have a on-board GPU to run the models. For this, the Raspberry Pi family \autocite{raspberrypi} was discarded due to the limitation of not having an onboard GPU. This leads to only be able to use a product of the NVIDIA Jetson family \autocite{nvidiaJetsonModules}, as this are on board computers designed to be used with \gls{ml} models.

For this project, the NVIDIA Jetson Nano of 4 GBs was used as it offered the best price performance ratio and was sufficient enough to perform the computing requirements.

\subsection{Networking}

In order to provide conectivity from the on-board computer to the server, a networking connection is needed to be established. As this deployment must be self sufficient and not depend on the infrastructure of each community, a router with 4G connection is deployed. For this case, a conventional router is used, the \todo{add router}. Moreover, in order to provide the connectivity for the different cameras and the computer, a switch was used. The switch was the \todo{add switch}.

\todo{add photo of switch and router}

\subsection{Rele Extension board}

In order to trigger the mechanism to activate the opening of the garage door of each community, this is done by means of a rele connected in paralel to the door system. The opening doors function with a line where a pull-down signal is sent whenever the mechanism wants to be activated. In order to perform this operation, a Rele Extension board was used for the NVIDIA Jetson. The reason is that after trials, the on-board GPIO were not strong enought to send a constant signal for some specific communities and the mechanism malfunctioned.

The Rele Extension board used was the \todo{add extension} due to the low price-point and the easy access to buy.

\section{Server}

The server is the point of contact to connect the users and the different communities. It allows the users to authenticate and connect to the sepcific community. It is based on a scalable archtutecture deployed in \gls{aws} due to the fast time-to-market capabilities and the scalability of it.

The server also hosts the website that the users use to connect to
\todo{write}

\section{Monitoring}

\todo{write}

\section{Deployment}

\todo{write}


\chapter{Implementation}\label{ch:implementation}

Given the requirements and the design in the previous chapters, refer to \cref{ch:requirements} and \cref{ch:design}, the project is implemented following the V-model established in \cref{ch:methodology_approach}. This chapter describes the implementation of the system, from the architecture to the deployment of the system.

\section{General Architecture}

The architecture, as seen in \cref{fig:architecture}, is composed of two main parts, the server and the communities. The server is in charge displaying the information to the users and the managers and handling the connection between the different communities. The communities are in charge of the detection of the vehicles, the opening of the gates, and the administration of the community (data storage, backup, etc).

\begin{figure}
	\includegraphics{architecture.png}
	\caption{General Architecture of the system}\label{fig:architecture}
\end{figure}

It is worth noting that the system is design to be scalable and robust, so the system is design to be able to be deployed in different communities and be able to handle the different data and users. That is why in \cref{fig:architecture} multiple communities can be seen.

\section{Community}

The community is a collection of the systems to provide the ability to open the gates, detect the vehicles, and store the data for a single building or community. A community is usually composed of three main subsystems, the licence plate detection system, the terminal server, and the actuators. Moreover, the community is connected to the internet to be able to connect to the server and perform other actions such as updates, remote administration, etc.

\subsection{Licence Plate Detection System}

The first component of the sytems is the detection of the licence plates. The detection of the licence plates is done by a deep learning algorithm that detects the licence plates in the images. The detection algorithm is based on YOLO \autocite{yolov8_ultralytics} and is implemented in the terminal. The detection algorithm is in charge of detecting the licence plates in the images and sending the data to the database.

The implementation of the algorithm is out of the scope of this project, and it is based on the Dual Licence Plate Recognition system \autocite{RAMAJOBALLESTER2024104608}. The system is composed of two main steps, the detection of the licence plate and the detection of the characters of the licence plate. Some examples can be seen in \cref{fig:licence_plate_detection}.

\begin{figure}
	\begin{subfigure}{0.95\textwidth}
		\includegraphics{licence_plate_recognition.jpg}
		\caption{Testing of the Licence Plate Detection System. \autocite{RAMAJOBALLESTER2024104608}}
	\end{subfigure}
	\br
	\begin{subfigure}{0.95\textwidth}
		\includegraphics{ocr_licence_plate.jpg}
		\caption{Testing of the OCR character detection of the Licence Plate Detection System. \autocite{RAMAJOBALLESTER2024104608}}
	\end{subfigure}

	\caption{Licence Plate Detection alghorithms. \autocite{RAMAJOBALLESTER2024104608}}\label{fig:licence_plate_detection}
\end{figure}

\subsection{Terminal Server}

To provide the main functionality of the system, a terminal server is used. The terminal server is in charge of the communication between the different subsystems of the community and the server. The terminal server is in charge of communicating with the server, the actuators, the detection algorithm, the data storage, the backup of the data, and the user administration.

For reference, the terminal server is designed ot be run on any Linux system, and the implementation is based on the ExpressJS framework. This makes the terminal server easy to deploy and maintain. Moreover, the terminal server is design to be able to be updated remotely and be able to be connected to the internet.

For the communication between the different subsystems, the terminal server uses a RestAPI methodology. This allows the different subsystems to communicate with the terminal server and perform the different actions.

\subsubsection{Database}

Different architecture approaches were considered for the database, such as SQL or NoSQL databases, local or cloud databases. For this project, a local SQL database based on SQLite is used. The main reason behind this decision was the requirement of having the data inside the community for data privacy and security reasons. The SQL architecture is used as it provides a robust and reliable way to store the data and be able to perform the different actions required such as filtering, searching, and more. The database architecture can be seen in \cref{fig:database_architecture}.

\begin{figure}
	\includegraphics{database_architecture.png}
	\caption{Database Architecture}\label{fig:database_architecture}
\end{figure}

The database is divided into four main tables to store the different types of data.

The first table is the \texttt{users} table. This table is in charge of storing the information of the users of the system. The table is composed of the following fields:

\begin{itemize}
	\item \texttt{id}: The unique identifier of the user which is a autoincrement numeric field.
	\item \texttt{name}: The name of the user to identify it.
	\item \texttt{prefix}: The prefix of the phone used by the user, as the users come from different countries.
	\item \texttt{phone}: The phone number of the user.
	\item \texttt{type}: The type of the user, which can be either an administator, an owner of a property, or a visitor.
	\item \texttt{nif}: The NIF of the user, which is used to identify the user for regulatory and security reasons.
	\item \texttt{property\_id}: The identifier of the property of the user.
	\item \texttt{lots}: The number of parking spaces that the user has.
	\item \texttt{owner\_id}: Id of the owner of the property if the user is a visitor.
	\item \texttt{terms\_accepted}: A boolean field to check if the user has accepted the terms and conditions of the system.
	\item \texttt{is\_pending\_for\_invite}: A boolean field to check if the user is pending for an invitation to the system.
\end{itemize}

The primary key of the table is the \texttt{id} field. And a foreign key is used to connect the \texttt{owner\_id} field to the \texttt{id} field between an owner and a visitor. Moreover, different check are performed to ensure the data integrity and the security of the data, for example, the \texttt{phone} field is checked to be a valid phone number and unique.

Secondly, the \texttt{plates} table is used to store the information of the licence plates registed in the system. The table is composed of the following fields:

\begin{itemize}
	\item \texttt{plate}: Alphanumeric field to store the licence plate, the length is variable to adapt to different countries.
	\item \texttt{name}: Identifier of the licence plate for easy identification.
	\item \texttt{is\_enabled}: Boolean field to check if the licence plate is enabled or disabled, that is, if it is allowed to enter the community.
	\item \texttt{is\_inside}: Boolean field to check if the licence plate is inside the community in order to have a control of the vehicles inside the community.
	\item \texttt{user\_id}: The identifier of the user that owns the licence plate.
\end{itemize}

The primery key of the table is the \texttt{plate} field. A foreign key is used to connect the \texttt{user\_id} field to the \texttt{id} field of the \texttt{users} table. Moreover, the \texttt{plate} field is checked to be unique and the \texttt{user\_id} field is checked to be a valid user.

For the configuration of the community, the \texttt{configuration} table is used. The table is composed of the following fields:

\begin{itemize}
	\item \texttt{id}: The unique identifier of the configuration. As there is only one configuration, the field is a constant and set to 1 by default.
	\item \texttt{name}: The identifier name of the community.
	\item \texttt{lots\_limit}: The maximum number of parking spaces in the community.
	\item \texttt{cameras}: The configuration of the cameras in the community. It is used to match the specific cameras with the specific actuators of the community.
	\item \texttt{doors}: The number of garage doors in the community as multiple doors can be used for different points of the community.
	\item \texttt{is\_button\_enabled}: Boolean field to check if the button to open the door is enabled to be able to open the door manually.
	\item \texttt{time\_wait\_enter\_in\_seconds}: Timeout to wait for a vehicle to enter the community.
	\item \texttt{time\_wait\_exit\_in\_seconds}: Timeout to wait for a vehicle to exit the community.
	\item \texttt{time\_max\_no\_authorized\_plate\_in\_seconds}: Timeout to wait for a vehicle to enter the community if the licence plate is not authorized.
	\item \texttt{date\_time\_discrimination}: Variable to check if the system is using the date and time to discriminate the vehicles.
	\item \texttt{yolo\_resolution}: Configuration for the Licence Plate Detection System.
\end{itemize}

The primary key is the \texttt{id} field and it is set to 1 in order to only have one configuration. This information is kept in the database to centralized everything and make it easier to backup the data. A separate file can be used for the configuration, but it is stored in the database for security reasons.

Finally, the \texttt{records} table is used to store the logs of the system, that is all the different interactions with the different subsystems, such as the licence plates detected, the users that enter the community, the users that exit the community, etc. The table is composed of the following fields:

\begin{itemize}
	\item \texttt{date}: The timestamp of the record.
	\item \texttt{plate}: The licence plate of the vehicle that is detected.
	\item \texttt{owner\_name}: The name of the owner of the vehicle that is detected to have a better control of access flow.
	\item \texttt{action}: The action performed, that is, if the vehicle is entering or exiting the community.
	\item \texttt{was\_allowed}: Boolean field to check if the vehicle was allowed to enter the community.
\end{itemize}

The primary key of the table is the \texttt{date} field, this is done to restrict multiple different records for a sigle timestamp. Moreover, the \texttt{date} field is set to use the ISO 8601 format to have a standard format for the date and time.

As the terminal is situated in the community, unauthorized access to the database is a concern. To keep the database secured from unauthorized access, the database is encrypted using the SQLCipher library. This library is used to encrypt the database and ensure that the data is secure and only accessible by the system. Keeping the database encrypted ensures that the data is secure and only accessible by the system.

\subsubsection{Application Programming Interface}

As stated previously, in order to communicate with the different subsystems, a RestAPI is used. The RestAPI is used to perform the different actions such as opening the door, detecting the licence plates, storing the data, etc. The RestAPI is built using the ExpressJS framework, as it is easy to use and maintain. For this different endpoints are used to perform the different actions such as user administration, licence plate detection, door opening, etc. In order to perform the authenicaton of the users, JWT tokens are used to ensure that the request is performed by the right user. The authenicaton process is later outlined in \cref{sec:authentication}.

\subsubsection{Actuators}

To interact with the various subsystems of the communities, particularly for controlling access points like garage doors, the system employs a custom Relay Expansion Board designed for the NVIDIA Jetson Nano. This board serves as the interface between the digital control signals from the terminal server and the physical actuators. The choice of this solution came after studying different door opening systems, all of which required a signal to be pulled down to ground to trigger the mechanism.

The control of doors is achieved through relays, which act as electrically operated switches. To enable both automated and on-demand operation of the doors, a small server is set up within the terminal. This server listens for requests to open the door, which can come from either the license plate recognition system or authorized user commands.

When a valid request is received by the server, it processes the command and sends an appropriate signal to the Relay Expansion Board. The board then activates the corresponding relay, which in turn triggers the door mechanism to open. This setup allows for seamless integration of the automated license plate recognition system with manual override capabilities, ensuring flexibility and reliability in access control.

The use of relays provides several advantages. It offers electrical isolation between the control circuitry and the door mechanisms, allows for the control of high-voltage or high-current devices with low-voltage signals, and provides flexibility to adapt to various types of door actuators used in different communities.

This actuator subsystem is crucial in translating the software-based decisions of the parking management system into physical actions, effectively bridging the digital and physical aspects of the solution. The system's ability to open doors based on both automated detection and manual requests ensures a versatile and user-friendly operation, catering to different scenarios and user needs within the parking management ecosystem.

\subsubsection{Other subsystems}

The system incorporates several additional components that work together to ensure reliable operation and data integrity. The surveillance infrastructure utilizes a network of cameras strategically positioned throughout the community, particularly at entry and exit points. Each camera captures video feeds and transmits them to the terminal for processing using the Real-Time Streaming Protocol (RTSP). The cameras connect to the network infrastructure via Ethernet cables, ensuring stable and high-quality video transmission.

Network connectivity is managed through a dedicated 4G router equipped with an integrated switch. This router, specifically the TP-Link AC1200 model, serves two critical functions: it facilitates communication between the cameras and the terminal, and provides internet connectivity for the entire system. The choice of a 4G router ensures that the system remains independent of the community's existing infrastructure while maintaining reliable connectivity. The integrated switch allows for efficient connection of multiple cameras and the terminal, simplifying the network topology and reducing potential points of failure.

To ensure data integrity and business continuity, the system implements a comprehensive backup strategy. The database, which contains critical operational data including user information and access logs, is automatically backed up daily to Amazon S3 \autocite{AmazonS3}. These backups are encrypted both during transmission and storage, protecting sensitive information from unauthorized access. This automated backup system ensures a \gls{rpo} of 24 hours, minimizing potential data loss in the event of system failure or data corruption. The combination of local encrypted storage and cloud-based backups provides a robust solution for data protection and recovery.

\section{Cloud deployment}

To enable the access to the different communities from the user point of view, a web server is deployed in AWS. The main component is a NextJS server that holds the website that the users use to access the system and also the backend functionality to connect to each specific community.

\subsection{Cloud Architecture}

The cloud architecture developed for this project adopts a cost-effective and scalable methodology, leveraging Amazon Web Services (AWS) as the primary cloud provider. As illustrated in \cref{fig:cloud_architecture}, the architecture employs a distributed design that prioritizes reliability and security while maintaining operational efficiency.

\begin{figure}
	\includegraphics{cloud_architecture.png}
	\caption{AWS Cloud Architecture} \label{fig:cloud_architecture}
\end{figure}

During the development phase, several architectural approaches were evaluated, including serverless computing, container-based solutions, and traditional virtual machine deployments. While serverless and container architectures offered certain advantages in terms of scalability and maintenance, the specific requirements of the networking component, particularly the integration with ZeroTier, necessitated the use of EC2 instances. This decision was primarily driven by ZeroTier's requirement for low-level network access and administrative privileges, which are not readily available in more abstracted deployment models.

The architecture is implemented within AWS's Virtual Private Cloud (VPC), which provides network isolation and security. Public and private subnets are utilized to create a layered security approach, with the public subnet hosting load balancers and other internet-facing components, while the private subnet contains the application servers and sensitive resources. This separation ensures that critical system components remain protected from direct external access.

To ensure high availability and fault tolerance, the EC2 instances are distributed across multiple Availability Zones within the chosen AWS region. An Auto Scaling group manages these instances, automatically adjusting capacity based on demand while maintaining optimal performance and cost efficiency. This approach allows the system to handle varying loads while minimizing operational expenses.

The networking layer is particularly crucial in this architecture, as it must facilitate secure communication between the cloud infrastructure and the distributed community terminals. ZeroTier integration, later explained in \cref{sec:networking}, enables the creation of secure, encrypted tunnels between AWS resources and on-premises community systems, effectively establishing a hybrid cloud environment that maintains both security and performance.

This architectural design successfully balances the requirements for scalability, security, and cost-effectiveness, while providing the necessary infrastructure to support the parking management system's distributed nature. The use of EC2 instances, though more traditional than newer deployment options, proves to be the most suitable choice for meeting the specific networking and security requirements of the system.

\subsection{Website}

The website component of the parking management system is designed with a strong focus on user experience and accessibility across different devices. Built using the React framework \autocite{react}, the website provides a modern, interactive interface that enables users to manage their parking spaces and access system features efficiently. React's component-based architecture facilitates the development of reusable interface elements while ensuring optimal performance through its virtual DOM implementation.

For styling and visual presentation, the website implements TailwindCSS \autocite{tailwindcss}, a utility-first CSS framework that enables rapid development of custom, responsive designs. This approach allows for consistent styling across different screen sizes while maintaining a clean, modern aesthetic that aligns with contemporary web design standards. The combination of React and TailwindCSS creates a robust foundation for building an intuitive user interface that adapts seamlessly to various device sizes and orientations.

An example of the website can be seen in \cref{fig:website_example}. The capabilities of the website are login, register, manage the different users, manage the different parking spaces, and manage the different configurations of the community. Moreover, the different records can be seen by the administrators.

The responsive design implementation ensures that the website functions effectively on both desktop computers and mobile devices, with layouts and interactions optimized for touch interfaces when accessed on smartphones or tablets. This adaptability is crucial for users who need to access the parking management system while on the move, particularly for tasks such as remotely opening garage doors or managing visitor access.

To enhance accessibility and provide a more native-like experience on mobile devices, the website is also implemented as a Progressive Web App (PWA). This implementation allows users to install the website as a standalone application on their Android or iOS devices directly from their respective app stores. The PWA approach combines the best aspects of web and native applications, offering features such as offline functionality, push notifications, and improved performance through local caching mechanisms.

The PWA implementation follows modern web standards and best practices, ensuring compatibility across different mobile platforms while maintaining a consistent user experience. This approach eliminates the need for platform-specific native applications while still providing the convenience and functionality users expect from a mobile app. Users can access features such as real-time parking space monitoring, license plate management, and access control directly from their home screens, making the system more accessible and user-friendly.

Moreover, the website is embeded in Progressive Web Apps in Android and iOS to allow users to download an application through the mobile store of choice and be able to use the website in their phones, \cref{fig:mobile_app}. It is worth noting that the application is a wrapper of the website, but it allows the users to have a more native experience. For the iOS version, the application is not available in the App Store, as the App Store has a more strict policy for the applications and it is currently under review.

\begin{figure}
	\hfill
	\begin{subfigure}{0.45\textwidth}
		\includegraphics{website_example.png}
		\caption{Website Example.}\label{fig:website_example}
	\end{subfigure}
	\hfill
	\begin{subfigure}{0.45\textwidth}
		\includegraphics{mobile_app.png}
		\caption{Android Applciation deployed in Google Play Store}\label{fig:mobile_app}
	\end{subfigure}
	\hfill

	\caption{Website design and mobile application.}
\end{figure}

\subsection{Networking}\label{sec:networking}

The interconnection of diverse communities with the web server in a decentralized and user-friendly manner is achieved through the implementation of ZeroTier \autocite{zerotier2025}. This innovative Wide LAN technology facilitates the connection of various devices across the internet by establishing secure VPN tunnels. ZeroTier proves to be an excellent solution for linking different communities to the web server, enabling a range of functionalities such as door access control and license plate recognition.

One of the primary motivations for utilizing EC2 instances in the cloud deployment was the seamless integration of ZeroTier. This approach eliminates the need for proxies or other intermediary measures, streamlining the overall system architecture. The direct installation of ZeroTier alongside the server on EC2 instances ensures a robust and efficient networking solution.

Furthermore, ZeroTier offers the invaluable capability of remote control access. This feature significantly enhances the ability to debug and troubleshoot issues within different communities without the necessity of on-site presence. Technicians and administrators can remotely access the system, diagnose problems, and implement solutions, greatly reducing response times and improving overall system maintenance.

The decentralized nature of ZeroTier aligns well with the project's goals of creating a flexible and scalable network infrastructure. As new communities are added to the system, they can be easily integrated into the existing network topology without major reconfiguration. This scalability ensures that the system can grow organically as more communities adopt the technology.

Security is another crucial aspect addressed by the ZeroTier implementation. The VPN tunnels created by ZeroTier provide encrypted communication channels, safeguarding sensitive data transmitted between the communities and the central web server. This encryption is particularly important when dealing with access control systems and personal information associated with license plate recognition.

In summary, the adoption of ZeroTier as the networking solution for this project offers a powerful combination of ease of use, scalability, remote management capabilities, and enhanced security. These features collectively contribute to a robust and efficient system that can reliably serve multiple communities while remaining adaptable to future growth and technological advancements.


\subsection{User Authentication}\label{sec:authentication}

The authentication system for this project implements a phone-based SMS verification approach, prioritizing both security and user experience. This decision was driven by the requirement to create a simple yet secure authentication process that would be easily accessible to all users, regardless of their technical expertise. Traditional password-based systems often lead to security vulnerabilities due to weak password choices or password reuse, making SMS-based authentication an attractive alternative.

Firebase Authentication was selected as the authentication provider for this implementation, offering a robust and scalable solution for handling SMS-based user verification. This service manages the entire SMS delivery and verification process, including phone number validation, SMS dispatch, and code verification. Firebase's global infrastructure ensures reliable message delivery across different cellular networks and geographical regions, which is crucial for a system deployed across multiple communities.

The authentication flow begins when a user attempts to access the system. They are prompted to enter their phone number, which is then validated for format correctness. Firebase Authentication sends a one-time verification code via SMS to the provided number. This temporary code must be entered by the user within a specified timeframe to complete the authentication process. Upon successful verification, Firebase generates a JSON Web Token (JWT) that serves as the user's authentication credential for subsequent interactions with the system, see \cref{fig:firebase} for reference.

These JWT tokens play a crucial role in maintaining security across the distributed architecture. When a user makes requests to either the cloud server or the terminal servers in specific communities, the JWT token is included in the request headers. The receiving systems verify the token's authenticity using Firebase's public keys, ensuring that only requests from properly authenticated users are processed. The tokens contain encoded information about the user's permissions and access levels, allowing the system to enforce appropriate access controls without requiring additional database queries.

To enhance security further, the tokens are configured with appropriate expiration times, requiring periodic re-authentication. This helps mitigate the risk of token theft or unauthorized access. Additionally, the system maintains a blacklist of revoked tokens to handle cases where immediate access termination is required, such as when a user's privileges are revoked or suspicious activity is detected.

The entire authentication process is integrated seamlessly into both the web interface and mobile applications, providing a consistent user experience across all platforms. Error handling mechanisms are implemented to manage various edge cases, such as failed SMS delivery, network connectivity issues, or invalid verification codes, ensuring that users receive clear feedback and guidance throughout the authentication process.

\begin{figure}
	\includegraphics{firebase.png}
	\caption{Firebase Authentication Users example}\label{fig:firebase}
\end{figure}

\section{Deployment}

The deployment process for this distributed parking management system encompasses multiple components, including terminal servers in communities and cloud-based web servers. A standardized approach leveraging modern continuous integration and deployment practices ensures consistent and reliable software updates across all system components. The foundation of this deployment strategy relies on JavaScript frameworks, with applications built and managed through the NPM package manager. To automate the build and release process, a sophisticated GitHub Action pipeline triggers whenever a new release is created in the Git repository. An example can be seen in \cref{fig:github_actions} where web server can be built for the \texttt{0.3.2} version.

\begin{figure}
	\includegraphics[width=0.7\textwidth]{github_actions.png}
	\caption{Example of the GitHub Actions Pipeline}\label{fig:github_actions}
\end{figure}

The deployment pipeline is designed to handle the complexities of distributing software across various environments while maintaining system integrity and security. When a new version is tagged in the Git repository, the GitHub Actions workflow automatically initiates the build process, creating NPM packages with all necessary dependencies. These packages are then stored in GitHub Packages, providing a centralized and secure location for distribution to both community terminals and cloud servers.

\subsection{Terminal Deployment}

For community terminals, the deployment process requires special consideration due to their distributed nature and the critical importance of maintaining data integrity. The terminal software updates are managed through automated processes, but database migrations present unique challenges. Currently, database migrations are handled manually to ensure data integrity and prevent potential issues that could arise from automated migration processes. This conservative approach reflects the critical nature of the parking management data and the need to maintain system reliability.

While this manual database migration process is functional, it represents an area identified for future improvement. The development team is actively working on designing a robust automated migration system that can maintain data integrity while reducing the operational overhead of manual interventions. This enhancement will need to carefully balance automation with data safety, ensuring that no critical information is lost or corrupted during updates.

\subsection{Server Deployment}

The cloud infrastructure deployment leverages Infrastructure as Code principles through Terraform, enabling consistent and repeatable cloud resource provisioning. This approach allows the entire AWS infrastructure to be defined and managed through version-controlled code, significantly reducing the potential for configuration errors and ensuring deployment consistency across different environments.

A key component of the server deployment strategy is the use of a Baked Amazon Machine Image (AMI) for EC2 instances within the auto-scaling group. This AMI contains all necessary dependencies and application code pre-installed, enabling rapid and reliable instance deployment. When new instances are launched, either during scaling events or updates, they begin with a consistent, production-ready configuration. This approach significantly reduces instance startup time and ensures uniformity across all deployed servers.

\section{Other Considerations}

The implementation of this parking management system necessitates careful consideration of legal and regulatory requirements, particularly those outlined in \cref{ch:regulatory_framework}. To ensure compliance with these regulations, especially the \gls{gdpr}, a comprehensive terms of service and privacy policy framework has been implemented.

The system requires explicit user consent before any personal data collection or processing occurs. This is implemented through a mandatory terms and conditions acceptance process during user registration. The terms of service clearly outline the system's data collection practices, user rights, and privacy protections, ensuring transparency in how personal information is handled. These terms are presented in clear, accessible language to ensure users can make informed decisions about their data.

Data minimization principles are strictly enforced throughout the system's operation. Only essential information required for the parking management functionality is collected and stored. This includes license plate numbers, basic user identification details, and access logs. The system maintains detailed records of user consent and data processing activities, as required by GDPR Article 30, which can be audited to demonstrate compliance with regulatory requirements.

To address the requirements of the ePrivacy Directive and the emerging AI Act, the system implements strict protocols for data retention and processing. All collected data is stored securely with encryption at rest and in transit. The retention period for personal data is clearly defined, and automated processes ensure that data is deleted when it is no longer necessary for the system's operation. This approach aligns with both the data minimization principle and the specific requirements for data retention outlined in the regulatory framework.

Regular updates to the terms of service are conducted to ensure ongoing compliance with evolving regulations and to address new requirements as they emerge. Users are notified of any significant changes to the terms and, where necessary, asked to provide renewed consent. This dynamic approach to legal compliance ensures that the system remains current with regulatory requirements while maintaining transparency with its users.




\oldpart{Results}\label{part:results}

\chapter{Testing}\label{ch:testing}

\todo{write this chapter}

add the problem with bad connection and trying different sims

\chapter{Deployment}\label{ch:deployment}

The deployment phase of this project involved the successful installation and integration of the parking management system across ten different communities in Valencia. This chapter details the deployment process, challenges encountered, and solutions implemented to ensure system reliability across diverse environments. 

\section{Deployment Process}

The deployment of new communities follows a standardized four-phase approach to ensure consistency and reliability. The initial phase involves physical hardware installation, including the NVIDIA Jetson Nano, cameras, networking equipment, and relay systems. Careful consideration is given to equipment placement, particularly for cameras, to optimize license plate recognition while protecting hardware from environmental factors. 

The second phase focuses on establishing network connectivity and integrating the community into the broader system architecture. This includes configuring the 4G router, setting up ZeroTier connections, and verifying communication with the central server. Each community receives a unique identifier within the network to maintain proper isolation and security. 

The third phase encompasses data population and system configuration through the web interface. This involves adding authorized users, registering license plates, and configuring community-specific parameters such as parking space limits and access rules. Administrator training is conducted during this phase to ensure proper system management. 

The final phase involves system validation and monitoring. During this period, the system operates under close supervision to verify all components function correctly and to address any community-specific requirements or issues that arise.

\section{Environmental Challenges}

Deployment across multiple communities revealed significant environmental challenges that impacted system reliability. The Dana weather event in Valencia in September 2024 \autocite{CNNSpainFlooding2024} severely affected three communities, where flooding damaged terminal equipment and disrupted system operations, see \cref{fig:dana}. This experience led to the implementation of enhanced waterproofing measures and the elevation of critical hardware components above potential flood levels. 

\begin{figure}
        \includegraphics{dana.jpg}
    \caption{Dana weather event in one of the communities in Valencia, September 2024}\label{fig:dana}
\end{figure}

Camera performance was notably affected by varying weather conditions. Direct sunlight caused glare issues that impacted license plate recognition accuracy, while heavy rain sometimes triggered false readings as water made the lense blurry \cref{fig:wet_camera}. These challenges were addressed through the installation of protective shields and the refinement of the recognition algorithms to account for adverse weather conditions. 

\begin{figure}
        \includegraphics{empanado.jpg}
    \caption{Water leakage inside a camera making the image blurry}\label{fig:wet_camera}
\end{figure}

\section{Community-Specific Adaptations}

Different communities presented unique infrastructure requirements that necessitated system adaptations. Several communities required dual garage door configurations to manage separate entry and exit points. This led to the development of enhanced traffic flow management algorithms and modified hardware setups to coordinate multiple access points effectively. 

The installation process revealed varying electrical infrastructure across communities, requiring custom solutions for power delivery and surge protection. In older buildings, existing wiring needed to be carefully evaluated and sometimes upgraded to support the system's power requirements while maintaining safety standards.




\oldpart{Conclusions}\label{part:conclusions}

\chapter{Conclusions}\label{ch:conclusions}

While this thesis presentes the design, implementation, and testing of a distributed parking management system tailored for multi-community deployment, it establishes the basic architecture to be applied to other \gls{iot} solutions with edge computing and storage requirements. The project addresses limitations in existing systems by focusing on scalability, reliability, and adaptability, aligning with the goals of next-generation smart cities. 

Following the methodology outlined in \cref{ch:implementation}, the system was successfully deployed in ten communities around Valencia. The deployment has been running for six months. The system has processed thousands of entries and exits and has proven to be stable and reliable. 

Throughout the deployment, the system has undergone updates to meet specific community requirements. A configuration page was implemented to allow community administrators to manage settings such as parking limits, access permissions, and camera configurations. ML detection was improved to adapt to changes in environmental factors and improve the reliability of the system. A mobile app was released with basic functionalities to the users to interact with the system. A backup system was built and deployed to protect the database of the community to prevent data loss. A system to remotely update the different terminals has been developed. 

The implemented system demonstrated the feasibility of a distributed architecture for parking management, capable of operating autonomously within individual communities while maintaining centralized monitoring and control. The results of the project demonstrate the potential of the system to optimize parking space utilization, reduce traffic congestion, and enhance the overall urban experience.

\chapter{Future work}\label{ch:future_work}

\todo{write this chapter}

add that to change the db to a nosql for the users and keep the sql for the records

use prisma for the db

\chapter{Socio-economic environment}\label{ch:socio_economic_environment}

The widespread adoption of automated parking management systems is transforming urban environments and the economy, offering significant benefits while also presenting challenges. This chapter examines the socio-economic factors influencing the deployment of these systems, particularly in residential communities. It explores the potential advantages, obstacles, and financial implications of integrating this technology across urban areas.

\section{Social Impact}

Automated parking management systems have gained popularity due to their efficiency and ability to optimize space utilization, revolutionizing urban planning and mobility. Their capacity to reduce traffic congestion and improve parking availability can greatly enhance urban living conditions. For example, sensor-based systems can guide drivers to available parking spots, reducing time spent searching and lowering emissions. The project demonstrated how these systems can provide real-time occupancy data, aiding both residents and visitors in finding parking more efficiently.

In densely populated urban areas, automated parking systems facilitate better use of limited space. This capability not only improves the quality of life for residents but also contributes to more sustainable urban development by maximizing the use of available land.

However, privacy and data security remain concerns. The extensive data collection by these systems raises issues of surveillance and personal data protection, underscoring the need for clear regulations and ethical guidelines, as discussed in the regulatory framework chapter.

\section{Economic Impact}

The economic benefits of adopting automated parking management systems are substantial, with the potential to boost urban efficiency and drive innovation. For property managers and urban planners, these systems enable more efficient use of space, potentially increasing property values and reducing operational costs associated with parking management.

Despite these advantages, the costs of implementing automated parking systems (e.g., installation, training, maintenance, and upgrades) remain significant barriers for some communities and smaller property management companies. Moreover, the need for technical expertise and the risk of system failures can increase operational expenses.

\section{Environmental Impact}

Automated parking management systems can positively impact the environment by reducing traffic congestion and associated emissions. By guiding drivers directly to available spots, these systems minimize the time vehicles spend idling or circling in search of parking, contributing to lower air pollution and fuel consumption in urban areas.

Nonetheless, the environmental costs of manufacturing and disposing of electronic components must be considered. The production of sensors, cameras, and other hardware components can be resource-intensive. Sustainable design practices and proper e-waste management are necessary to mitigate these effects.

\section{Planning}

The project was structured into distinct phases, each with specific activities and tasks, as illustrated in \cref{fig:planning}. The timeline for these phases was planned on a weekly basis, detailing the progression of activities from initial research to final deployment. The first month and a half of the project was dedicated to understanding the problem, reviewing relevant literature, exploring the necessary tools, and configuring them for our specific needs. This initial phase, encompassing research and planning, required a significant time investment due to the complexities and documentation gaps associated with certain tools like AWS or hardware integration. The subsequent five months were focused on development of the system and the final three months to writing the final document.

\begin{figure}
    \includegraphics{planning.png}
    \caption{Gantt Chart for the Project Planning}\label{fig:planning}
\end{figure}

\section{Budget Analysis}

The cost of implementing automated parking management systems is a significant consideration for many communities. The analysis includes a detailed assessment of hardware components, software, maintenance, and other relevant expenses.

\cref{tab:hardware_costs_physical_components} details hardware costs for terminals, cameras, and networking equipment costs and other expenses for each community. The total cost for these physical components is estimated at 621 \euro\ per community. However, it is worth noting that for this project, 10 different communiteis have been installed totaling to 6210 \euro\ for the physical components. Moreover, in \cref{tab:costs_software_components} the cost of the software components are included totaling to 30 \euro. Moreover, the human costs associated to this project, refer to \cref{tab:human_costs}, amount to 13900 \euro.

For this project, the total cost would amount to 20140 \euro, which includes the necessary hardware and software for a typical residential community and the human labor costs. However, it's important to note that this does not account for installation and long-term maintenance costs or potential upgrades, which are essential for the system's longevity and effectiveness.

In conclusion, while automated parking management systems hold significant promise for urban areas and residential communities, careful consideration of social, economic, and environmental factors is essential. Addressing these challenges with appropriate regulations and sustainable practices will maximize their potential benefits in creating more livable and efficient urban spaces.

\begin{table}[H]
	\begin{tabular}{ l l r r }
		\toprule
		\textbf{Item}               & \textbf{Model}                                & \textbf{Quantity} & \textbf{Cost (\euro)} \\
		\midrule
		Edge Computer               & NVIDIA Jetson Nano \autocite{reComputerJ1010} & 1                 & 220                   \\
		Cameras                     & Reolink RLC-811A \autocite{ReolinkRLC811A}    & 2                 & 166                   \\
		Router                      & TP-Link AC1200 \autocite{TPLinkArcherMR600}   & 1                 & 85                    \\
		Reles                       &                                               & 1                 & 10                    \\
		Ethernet Cables             & Cat 6 30 meters                               & 1                 & 20                    \\
		4G SIM Card                 & Orange Prepaid SIM Card (1 month)             & 1                 & 10                    \\
		Boxes, power supplies, etc. &                                               & 1                 & 100                   \\
		\midrule
		\textbf{Total}              &                                               &                   & 621                   \\
		\bottomrule
	\end{tabular}
	\caption{Hardware costs for the physical components for one community.}\label{tab:hardware_costs_physical_components}
\end{table}

\begin{table}[H]
	\begin{tabular}{ l l l r }
		\toprule
		\textbf{Item}  & \textbf{Model}                           & \textbf{Quantity} & \textbf{Cost (\euro)} \\
		\midrule
		Server         & Amazon Web Services Deployment (1 month) & 1                 & 30                    \\
		\midrule
		\textbf{Total} &                                          &                   & 30                    \\
		\bottomrule
	\end{tabular}
	\caption{Costs for the software components and maintenance.}\label{tab:costs_software_components}
\end{table}

\begin{table}[H]
    \begin{tabular}{ l l l r }
        \toprule
		\textbf{Item}  & \textbf{Model}                           & \textbf{Quantity} & \textbf{Cost (\euro)} \\
        \midrule
        Personal Computer & Macbook Air & 1 & 1600  \\
        Junior Engineer Hours & 15 \euro/h & 500 & 7500 \\
        Senior Engineer Hours & 60 \euro/h & 40 & 2400 \\
        Electricity, labs, climate control, management, etc. & & & 2040 \\
        \midrule
        \textbf{Total} & & & 13900 \\
        \bottomrule
    \end{tabular}
    \caption{Human costs.}\label{tab:human_costs}
\end{table}

\chapter{Regulatory Framework}\label{ch:regulatory_framework}

\todo{Write this chapter}


\blankpage%
\printbibliography%

\end{document}
