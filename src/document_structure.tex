\chapter{Document Structure}\label{ch:document_structure}

This thesis is organized into six main parts, each addressing specific aspects of developing a scalable parking management solution for multi-community deployment.

Theoretical Background, \cref{part:theoretical_background}, provides essential knowledge for understanding the technological foundations of the project. It explores cloud computing concepts, including deployment models and service types crucial for system scalability. This part also covers Internet of Things (IoT) technologies, focusing on their architecture, benefits, and challenges in modern applications.

State of the Art, \cref{part:state_of_the_art}, examines the evolution and current state of parking management systems. It traces the historical development of these systems from early mechanical solutions to modern digital implementations. The section on modern trends explores current technologies and approaches in parking management, providing context for the project's innovations.

Methodology, \cref{part:methodology}, forms the core of the thesis, detailing the system's development process. It begins with comprehensive requirements analysis, covering both user and system needs. The design section outlines the system architecture and hardware selections, while the implementation section details the practical realization of the system, including both community-level and cloud components.

Results, \cref{part:results}, presents the outcomes of the project through testing and deployment phases. The testing section covers individual component validation and integrated system testing, while the deployment section discusses the real-world implementation across multiple communities and the challenges encountered.

Conclusions, \cref{part:conclusions}, summarizes the project's achievements and insights. It includes future work recommendations. This part provides a comprehensive overview of the project's implications and potential future developments.
