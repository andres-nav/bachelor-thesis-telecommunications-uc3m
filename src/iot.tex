\chapter{Internet of Things}\label{iot}

The \gls{iot} is a paradigm that establishes a network of interconnected devices equipped with sensors, software, and communication technologies. This enables seamless interaction between the physical and digital realms, facilitating the collection, transmission, and processing of data. Through its ability to deliver enhanced functionality and automation, the \gls{iot} has numerous applications, particularly in smart city infrastructure. These applications span healthcare, transportation, energy management, and urban planning.

The main point of \gls{iot} is its capacity for interconnectivity, enabling devices to autonomously communicate and share data. Advanced sensing technologies augment this connectivity, capturing real-world metrics such as temperature, motion, or occupancy. The processing of the data is achieved through edge computing, which provides localized analysis, or cloud computing for centralized processing, tailored to meet latency and scalability demands. This processed data is subsequently integrated with actuators, enabling automation of tasks such as the management of parking space access without requiring human intervention.

\section{Architecture of IoT Systems}
The architecture of \gls{iot} systems is structured across four principal layers. The perception layer constitutes sensors and actuators, which interface directly with the physical environment to gather data and execute specific actions. Data transmission occurs within the network layer, leveraging communication protocols such as \gls{WiFi}, \gls{Bluetooth}, or cellular networks. The processing layer transforms raw data into actionable insights through edge or cloud computing methodologies. Finally, the application layer provides interfaces for end-users, ensuring an intuitive and accessible experience tailored to diverse functionalities.

\section{Benefits of IoT}
The adoption of \gls{iot} technologies confers significant advantages across multiple sectors, particularly in the realm of parking management systems. Real-time monitoring capabilities, enabled by sensors, deliver continuous updates on parking space availability, reducing search times and alleviating urban congestion. Automation, facilitated by actuators, streamlines operations such as access control and reservation management, minimizing human intervention. Moreover, \gls{iot}-driven data analytics empower urban planners with insights to optimize resource allocation and develop informed policies. The energy efficiency inherent in smart systems aligns with sustainability objectives by curbing unnecessary resource consumption.

\section{Challenges and Limitations}
The deployment of \gls{iot} systems presents several challenges. Scalability remains a critical concern, particularly in large-scale implementations like city-wide parking systems. The management of sensitive information collected by \gls{iot} devices necessitates data security and privacy measures to itigate risks. Interoperability challenges, stemming from disparate manufacturer standards and communication protocols, complicate system integration.

