\chapter{Motivation}\label{ch:motivation}

The exponential growth of urban populations and the resulting increase in the number of vehicles have exacerbated the challenges of parking management in cities worldwide. For example, the significant surge in car usage in Madrid has had a profound impact on environmental pollution levels \autocite{environmental_impact_madrid_central}. Despite the stable number of parking spaces in Spain between 2014 and 2020 \autocite{urban_mobility_trends}, the increased demand for parking has led to higher costs, longer search times, traffic congestion, and elevated levels of urban pollution.

The ongoing development of \glspl{pms} remains a crucial research area, as traditional manual \glspl{pms} have proven inadequate in addressing the complexities of modern urban parking. Research indicates that the type of \gls{pms} employed significantly influences parking choices, with citizens favoring systems that prioritize user-friendliness, security, and reliability \autocite{parking_choices}.

Recent advancements in technology offer promising solutions to the limitations of traditional \gls{pms}. The proliferation of internet-connected devices, which has increased by 20\% year-over-year \autocite{iot_growth}, has facilitated the development of \gls{iot}-based \glspl{pms}. These systems aim to alleviate traffic congestion and reduce urban pollution, contributing to the implementation of sustainable and efficient urban mobility strategies.

The integration of modern \gls{pms} in urban settings provides numerous benefits, including the reduction of traffic congestion and pollution, enhancement of security, and improvement of the quality of life for residents. Furthermore, \gls{pms} offer the potential for automated parking space management, facilitating their application in diverse settings such as cities, communities, and buildings.

However, the widespread implementation of \gls{pms} in Spain is often hindered by the reliance on human intervention, resulting in issues such as delayed availability, inadequate information, and insufficient control of parking spaces. The continuous human presence required to maintain these systems results in increased operational costs.

Recent technological advancements have led to the development of innovative solutions, including RFID-based smart \glspl{pms} \autocite{rfid_smart_parking_management_system}, \gls{iot}-based smart \glspl{pms} \autocite{development_smart_parking_management_system}, and intelligent parking systems utilizing image processing \autocite{intelligent_parking_system_image_processing}. Despite these advancements, challenges persist, with current \glspl{pms} often featuring centralized architectures that compromise scalability and reliability in the event of service interruptions. Furthermore, these systems often lack customization options, hindering their adaptability to user-specific needs.

In light of these limitations, the primary objective of this project is to design and implement a fully distributed \gls{pms} that addresses the shortcomings of current systems, focusing on enhancing scalability, reliability, and user adaptability, with a particular emphasis on the requirements of next-generation smart cities.

% Local Variables:
% jinx-local-words: "iot rfid"
% End:
