\chapter{Motivation}\label{ch:motivation}

In recent years, the exponential increase in the number of vehicles has aggravated urban challenges, notably in parking management. For instance, the surge in car usage in Madrid has significantly contributed to an increase in pollution levels \autocite{environmental_imapct_madrid_central}. Despite a stable number of parking spaces in Spain between 2014 and 2020 \autocite{urban_mobility_trends}, the demand has led to higher costs and greater difficulty in securing parking, thereby increasing time spent searching for spaces, traffic congestion, and urban pollution.

Parking management systems remain a critical area of research and development, even though manual parking management has been utilized for many years. Research published in \autocite{parking_choices} indicates that the type of parking management system significantly influences the parking choices of citizens. Additionally, citizens tend to prefer systems that are user-friendly, secure, and reliable.

Advancements in technology offer promising solutions to these issues. The proliferation of internet-connected devices, which have grown by 20\% year-over-year \autocite{iot_growth}, has facilitated the development of Internet of Things (IoT)-based parking management systems. These systems aim to mitigate traffic congestion and reduce pollution in urban areas.

PMSs provide numerous benefits in modern urban settings. They contribute to reducing traffic congestion and pollution, thereby improving urban mobility and the quality of life for residents. Additionally, PMSs enhance security by monitoring vehicle movements within parking areas, which aids in crime prevention.

These systems share several common features:
\begin{itemize}
\item Automation of parking space management.
\item Application in various areas such as cities, communities, or buildings.
\item Documentation of vehicle transit within parking spaces.
\end{itemize}

Currently, parking management systems in Spain largely rely on human intervention, leading to several issues such as delays in availability, lack of information, and insufficient control of parking spaces. Continuous human presence is required to maintain the functionality of these systems throughout the day, which increases operational costs.

To address these issues, recent technological advancements have emerged. For instance:
\begin{itemize}
\item RFID-based smart parking management systems \autocite{rfid_smart_parking_management_system} have been developed to manage the transit of vehicles within parking spaces.
\item IoT-based smart parking management systems \autocite{development_smart_parking_management_system} have been designed to manage parking spaces within a community.
\item Intelligent parking systems utilizing image processing \autocite{intelligent_parking_system_image_processing} have been proposed to recognize license plates of parked vehicles.
\end{itemize}

Despite these advancements, several challenges persist. Current parking management systems are typically centralized, which presents scalability and reliability issues in the event of service interruptions. Additionally, these systems often lack customization options, making it difficult to adapt to user-specific needs.

Given these challenges, the primary objective of this project is to design and implement a fully distributed parking management system that addresses the limitations of current systems, focusing on enhancing scalability, reliability, and user adaptability.
