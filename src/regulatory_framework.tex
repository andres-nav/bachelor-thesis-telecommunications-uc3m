\chapter{Regulatory Framework}\label{ch:regulatory_framework}

The regulatory framework governing smart cities in Europe is primarily based on the data protection provisions outlined in the \gls{gdpr}. This is largely because these systems employ surveillance technologies such as cameras that monitor public spaces, and they handle sensitive personal data, including individuals' identification numbers and contact information. Additionally, the integration of technologies such as the \gls{iot}, \gls{ai}, and data analytics has necessitated robust legal mechanisms to address privacy, security, and ethical considerations.

\section{Applicable Legislation}

The implementation and operation of smart city technologies in the European Union must adhere to a range of legislative measures that regulate data privacy, electronic communications, and emerging technologies. Chief among these is the \gls{gdpr}, but other complementary regulations and directives are also critical in shaping the regulatory landscape.

\subsection{General Data Protection Regulation (GDPR)}

The \gls{gdpr} \autocite{gdpr}, also known as the Regulation (EU) 2016/679 and enacted in 2016, provides a comprehensive framework for the protection of personal data within the European Union. Its principles of transparency, accountability, and data minimization serve as the cornerstone of privacy and data protection.

Under \gls{gdpr}, smart city systems are required to obtain explicit consent from individuals before collecting or processing their personal data. Furthermore, data controllers must implement privacy-by-design and privacy-by-default principles, ensuring that data protection is an integral aspect of technological development. Smart cities must also conduct Data Protection Impact Assessments for projects involving high-risk data processing, such as large-scale surveillance or facial recognition.

\subsection{Data Retention Policies}

Smart city technologies often rely on continuous data collection through sensors, cameras, and connected devices. The \gls{gdpr} mandates that data should only be retained for as long as necessary to fulfill its intended purpose. To comply, smart city operators must establish robust data retention and deletion policies. Automated systems are frequently employed to enforce these policies, ensuring that unnecessary data is securely deleted to minimize privacy risks and reduce storage costs.

\subsection{ePrivacy Directive}

The ePrivacy Directive (Directive 2002/58/EC) \autocite{eu-58-2002}, commonly known as the "Cookie Law," governs the confidentiality of communications and the processing of electronic communications data. While initially designed for traditional telecommunications, its principles are increasingly relevant to smart city applications, such as real-time traffic monitoring and public Wi-Fi networks.

Under this directive, consent is required for the use of tracking technologies and location-based services, ensuring that users are aware of how their data is being utilized. The ongoing transition to the ePrivacy Regulation aims to enhance these protections and provide greater harmonization across member states.

\subsection{Directive on Security of Network and Information Systems (NIS 2 Directive)}

The \gls{nis} 2 Directive \autocite{eu-1148-2016}, adopted in 2016, focuses on the security of critical infrastructure and essential services, including those used in smart cities. It requires operators of essential services to implement risk management measures and report significant cybersecurity incidents to relevant authorities. For smart cities, this includes safeguarding critical systems such as transportation networks, energy grids, and communication systems against cyber threats.

\subsection{Artificial Intelligence Act}

The Artificial Intelligence Act \autocite{eu-1689-2024}, which came into force on 1 August 2024, establishes a common regulatory and legal framework for \gls{ai} across the European Union. Its provisions will be implemented gradually over the following 6 to 36 months. The Act classifies AI systems into categories based on risk, ranging from minimal to unacceptable, and imposes specific requirements on high-risk AI applications.

For smart cities, the Act directly impacts technologies such as AI-driven surveillance systems, autonomous transportation, and automated decision-making tools. These applications must adhere to stringent requirements, including transparency, accountability, and mandatory human oversight. The Act ensures that AI technologies in smart cities are deployed ethically and align with EU values, particularly with respect to fundamental rights and safety.

\section{Regulatory Challenges and Implications}

The evolving nature of smart city technologies presents significant challenges for regulatory compliance. These challenges include the integration of multiple technologies such as \gls{iot}, \gls{ai}, and data analytics, which often involve cross-border data flows and raise questions about jurisdiction and enforcement. Furthermore, the pace of technological innovation frequently outstrips the development of regulatory frameworks, creating gaps that may expose users to privacy and security risks.

Ensuring compliance with a wide array of legislative measures requires substantial investment in training and capacity-building for smart city administrators. Policymakers and industry stakeholders must collaborate to establish clear guidelines that address emerging issues while balancing innovation and the protection of fundamental rights.

