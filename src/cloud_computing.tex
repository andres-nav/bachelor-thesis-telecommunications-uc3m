\chapter{Cloud Computing}\label{ch:cloud_computing}

Cloud computing is a technological paradigm that enables the delivery of computing services, including servers, storage, databases, networking, software, and analytics, over the internet. This model shifts the management of physical infrastructure and resources from on-premises systems to remote data centers managed by cloud service providers. Cloud computing is fundamental to modern digital ecosystems, supporting a wide range of applications from personal use to large-scale enterprise operations.

\section{Definition and Key Concepts}

Cloud computing leverages a network of remote servers hosted on the internet to store, manage, and process data, rather than relying on local computers or private data centers. It offers resources on-demand, enabling users to scale operations according to their needs. The essential characteristics of cloud computing include:

\begin{itemize}
	\item \textbf{On-Demand Self-Service:} Users can access computing resources as needed without requiring human interaction with service providers.
	\item \textbf{Broad Network Access:} Services are accessible over a network, typically the internet, using various devices such as smartphones, tablets, and laptops.
	\item \textbf{Resource Pooling:} Resources are pooled to serve multiple users, ensuring efficiency and scalability.
	\item \textbf{Rapid Elasticity:} Resources can be elastically provisioned and released, enabling scalability according to demand.
	\item \textbf{Measured Service:} Cloud systems automatically control and optimize resource use, charging users based on their consumption.
\end{itemize}

\section{Advantages of Cloud Computing}

The adoption of cloud computing offers several advantages that have contributed to its widespread popularity. One of the most significant benefits is its scalability and flexibility. Cloud computing enables organizations to adjust their resource usage dynamically based on real-time needs, eliminating the need for over-provisioning and supporting workloads of varying demands effectively. This adaptability ensures that resources are utilized efficiently, minimizing waste and optimizing performance.

Cost efficiency is another critical advantage. By using cloud services, organizations can avoid the significant capital expenditures associated with purchasing and maintaining hardware and software. Instead, they incur operational costs only for the resources they consume, making cloud computing an economically viable solution for businesses of all sizes. Additionally, cloud computing enhances accessibility and collaboration. Since cloud-based applications and services are accessible from anywhere with an internet connection, distributed teams can collaborate seamlessly, a feature that has become indispensable in today’s increasingly globalized and remote work environments.

Reliability is also a key strength of cloud computing. Leading cloud providers ensure high levels of availability by deploying redundant systems across geographically dispersed data centers, thereby minimizing downtime and enhancing disaster recovery capabilities. Furthermore, security is a paramount concern addressed by cloud providers. They invest significantly in advanced measures such as encryption, firewalls, and regular security audits, ensuring robust protection against cyber threats and unauthorized access. These combined advantages make cloud computing a transformative technology, supporting innovation and operational efficiency in diverse sectors.

\section{Types of Cloud Computing}

Cloud computing services are categorized into different types based on their deployment models and the services they provide. These distinctions allow organizations to select a cloud strategy that aligns with their specific requirements.

\subsection{Deployment Models}

\begin{itemize}
	\item \textbf{Public Cloud:} A public cloud is owned and operated by third-party providers, delivering resources over the internet. Examples include \gls{aws}, Microsoft Azure, and Google Cloud Platform. It offers cost-effectiveness and scalability but may pose data sovereignty concerns for certain organizations.
	\item \textbf{Private Cloud:} A private cloud is used exclusively by a single organization. It offers greater control over data and infrastructure, making it suitable for industries with strict regulatory requirements.
	\item \textbf{Hybrid Cloud:} A hybrid cloud combines public and private cloud environments, enabling organizations to benefit from both scalability and control. This model is particularly useful for balancing workloads and maintaining sensitive data on-premises while leveraging the public cloud for other operations.
	\item \textbf{Multi-Cloud:} A multi-cloud strategy involves using multiple cloud providers to mitigate risks associated with vendor lock-in and enhance reliability and performance.
\end{itemize}

\subsection{Service Models}

Cloud computing services are typically offered in the following models:

\begin{itemize}
	\item \textbf{\gls{iaas}:} Provides virtualized computing resources over the internet, including servers, storage, and networking. Users manage operating systems and applications while the provider handles the infrastructure.
	\item \textbf{\gls{paas}:} Offers a platform that includes hardware, software, and development tools to build, test, and deploy applications. It abstracts the complexities of infrastructure management, enabling developers to focus on coding.
	\item \textbf{\gls{saas}:} Delivers software applications over the internet on a subscription basis. Users can access the software through a browser without worrying about installation or maintenance.
\end{itemize}
