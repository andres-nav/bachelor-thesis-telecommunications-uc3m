\chapter{Conclusions}\label{ch:conclusions}

This thesis presented the design, implementation, and testing of a distributed parking management system tailored for multi-community deployment. The project addressed limitations in existing systems by focusing on scalability, reliability, and adaptability, aligning with the goals of next-generation smart cities. 

Following the methodology outlined in \cref{ch:implementation}, the system was successfully deployed in ten communities around Valencia. The deployment has been running for six months. The system has processed thousands of entries and exits and has proven to be stable and reliable. 

Throughout the deployment, the system has undergone updates to meet specific community requirements. A configuration page was implemented to allow community administrators to manage settings such as parking limits, access permissions, and camera configurations. ML detection was improved to adapt to changes in environmental factors and improve the reliability of the system. A mobile app was released with basic functionalities to the users to interact with the system. A backup system was built and deployed to protect the database of the community to prevent data loss. A system to remotely update the different terminals has been developed. 

The implemented system demonstrated the feasibility of a distributed architecture for parking management, capable of operating autonomously within individual communities while maintaining centralized monitoring and control. The results of the project demonstrate the potential of the system to optimize parking space utilization, reduce traffic congestion, and enhance the overall urban experience.
